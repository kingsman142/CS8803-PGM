\textbf{Problem 5}

$p(d \vert s, D)\\
= \frac{p(d, s, D)}{p(s, D)} \\
= \frac{p(s, D \vert d)p(d)}{p(s, D)} \\
= \frac{p(s \vert d)p(D \vert d)p(d)}{p(s)p(D)} \\
= \int_{W, b, p} \frac{p(s, W, b, p \vert d)p(D, W, b, p \vert d)p(d, W, b, p)}{p(s, W, b, p)p(D, W, b, p)}$

We know $p(d, W, b, p) = p(d)p(W, b, p)$ (i.e. they are independent), as $W, b, p$ represent the parameters of the new data $D = \{s^n, d^n\}$, which is independent from already-existing data $d$. So, we get the following:

$= \int_{W, b, p} \frac{p(s, W, b, p \vert d)p(D, W, b, p \vert d)p(d)p(W, b, p)}{p(s, W, b, p)p(D, W, b, p)} \\
= \int_{W, b, p} \frac{p(s, W, b, p \vert d)p(d)p(D, W, b, p \vert d)p(W, b, p)}{p(s, W, b, p)p(D, W, b, p)} \\
= \int_{W, b, p} p(d \vert s, W, b, p)\frac{p(D, W, b, p \vert d)p(W, b, p)}{p(D, W, b, p)} \\
= \int_{W, b, p} p(d \vert s, W, b, p)\frac{p(W, b, p)p(D, W, b, p \vert d)}{p(D, W, b, p)} \\
= \int_{W, b, p} p(d \vert s, W, b, p)\frac{p(W, b, p)p(D, W, b, p)}{p(D, W, b, p)} \\
= \int_{W, b, p} p(d \vert s, W, b, p)\frac{p(W, b, p)p(W, b, p)p(D \vert W, b, p)}{p(D, W, b, p)} \\
= \int_{W, b, p} p(d \vert s, W, b, p)\frac{p(W, b, p)p(W, b, p)p(D \vert W, b, p)}{p(D)p(W, b, p)} \\
= \int_{W, b, p} p(d \vert s, W, b, p)\frac{p(W, b, p)p(D \vert W, b, p)}{p(D)} \\
= \int_{W, b, p} p(d \vert s, W, b, p)\frac{p(W, b, p)\prod_{n=1}^N p(s^n \vert d^n, W, b)p(d^n \vert p)}{p(D)} \\
= \int_{W, b, p} p(d \vert s, W, b, p)p(W, b, p \vert D) \qquad\qed \\$

In order to use sampling to estimate $p(d_i = 1 \vert s, D)$, we first need to use $p(W, b, p \vert D)$ to sample the set of parameters $W, b, p$. This is possible since there is data $D = \{s^n, d^n\}$ supplied to us. Now, once we have $W, b, p$ as fixed parameters, we can sample $p(d_i \vert s, W, b, p)$. As such, that initial sampling of the parameters allows us to sample a disease vector $d$. Then, we can proceed as we usually would with Gibbs sampling, initializing some disease vector $d$, fixing $d_{-i}$ for iteration $i$, and then sampling $d_i$. Over time, as we sample the values of $d_i$, we see how often $d_i = 0$ and $d_i = 1$ appear, and can compute $p(d_i = 1 \vert s, D) = \frac{\text{count}(d_i = 1)}{\text{count}(d_i = 0) + \text{count}(d_i = 1)}$ to get the marginals.